
\chapter{Final Considerations} \label{cap:concl}
%\vspace{-2mm}

%Este trabalho apenas demonstra a viabilidade da contagem automatizada de UFCs de bactérias em placas de petri. Sendo apenas o inicio para estudos futuros, desta forma este trabalho apresenta apenas uma das várias possibilidades abordagem possíveis para resolução  deste problema. De forma alguma este trabalho tem a intensão de se tornar conclusivo, determinando um ponto final no estudo deste assunto. Desta maneira as conclusões aqui apresentada refere-se apenas aos resultado apresentados, sem o intuito de encerrar o assunto, ao contrário, pretende-se dar inicio a uma discussão mais detalhada sobre o assunto em questão.

Foi possível definir a padronização na aquisição das imagens que formaram a base. Esta padronização se mostrou suficiente no que diz respeito a posição, distância, iluminação e posicionamento da câmera fotográfica. Essas características permitiram a definição de um algorítimo que se mostrou eficiente para localização da placa de petri em questão, obtendo uma pequena margem de erro em poucas placas, sem comprometer nenhuma placa em sua totalidade, inclusive se mostrando flexível quanto ao deslocamento da mesma dentro da imagem. Apesar de apresentar pequenas falhas que não comprometeram a identificação da placa e nem a contagem das UFCs de bactérias, foi possível observar que alguns cuidados no ato da aquisição da imagem podem evitar alguns problemas tais como: retirar a tampa da placa de petri, a presença da tampa ocasiona brilho excessivo nas bordas da placa de petri, fazendo com que a borda tenha valores de saturação semelhantes as UFCs de bactérias, dificultando assim a identificação das mesma nessa região; Outro cuidado refere-se ao fundo utilizado, que deve conter o mínimo possível de riscos ou sujeira, evitando assim que ruídos sejam adicionados as imagens. Estes ruídos podem interferir na contagem das UFCs de bactérias pequenas, causando falsos negativos em placas com UFCs de bactérias pequenas e falsos positivos em placas com pouca quantidade de UFCs de bactérias.

Para a contagem das UFCs de bactérias, foram estudadas técnicas de processamento de digital como, algoritmos de pré-processamento das imagens, para realçar pontos de interesse e minimizar os ruídos e técnicas de segmentação que propiciassem a separação de fundo e a identificação das UFCs de bactérias. Para o devido realce das áreas de interesse foram estudadas técnicas como \textit{blur}, \textit{gaussian blur}, filtro bilateral e filtro de mediana, tendo o último se mostrado mais útil devido ao caráter variável tanto das placas de petri bem como as próprias UFCs de bactérias, por eliminar uma boa quantidade de ruídos sem comprometer as UFCs de bactérias de tamanho reduzido em placas com alta densidade de UFCs de bactérias. Para a segmentação foram estudados espaços de cores, sendo identificado que os canais de cores da família H apresentam melhores resultados, se destacando o canal HLS. Além do espaço de cores também foram utilizadas técnicas de morfologia matemática tais como: erosão; dilatação; abertura; fechamento; \textit{top hat} e \textit{black hat}. Todas essas técnicas se mostraram úteis tanto na identificação da placa de petri como a própria identificação das UFCs de bactérias. E por fim foram estudas técnicas de limiarização, aqui vale destacar seu valor ambíguo, sendo igualmente usada na fase de realce quanto na de segmentação. Para a abordagem utilizada nesse trabalho tais técnicas se mostram suficiente, pois permitiram de modo satisfatório uma segmentação possível baseando-se na saturação e luminosidade das imagens. Porém é importante ressaltar que esse conjunto de técnicas não abrangem as diversas características das imagens, por tanto, a adição de outras técnicas pode complementar este trabalho.

Dentre as possíveis abordagem para resolução do problema proposto em questão, a abordagem baseada apenas em técnicas de processamento de imagens, centrando-se na saturação e brilho das imagem foi a que se mostrou mais viável, não sendo definitiva. Tomando a contagem manual como referência, foi possível observar que a contagem automática obteve uma forte correlação em comparação com a contagem manual, segundo o cálculo de \textit{pearson} apresentado o valor de 0.9486 e erro absoluto médio de 0.2243, levando em consideração apenas os tempos por horas (média dos logs). Nesse ponto é importante destacar que esse valor não está levando em consideração a margem de erro e também é necessário considerar a baixa quantidade amostral (apenas 16 tempos distintos). Essa ressalva fica ainda mais evidenciada quando é feita uma análise mais detalhada, comparando as contagens individualmente. Para esses casos os resultados demonstram uma queda significativa no cálculo de \textit{pearson}, evidenciando uma correlação bem inferior e o erro médio absoluto apresenta valores relativamente altos.

Com tudo, é importante destacar o desempenho do \textit{software} no que se refere ao tempo. Para que se obtenha a curva de crescimento do micro-organismo são necessárias inocular e incubar 45 placas em duplicatas, somando 90 placas a serem contadas. Desta forma um técnico analista de laboratório de microbiologia qualificado e experiente pode levar mais de 8 horas de trabalho, ou seja, mais de um dia, para realizar toda a contagem e todas as 90 placas. Para que pudesse ser executado pelo \textit{software} foi obtida duas fotos de cada placa, totalizando 180 imagens e utilizou-se um \textit{notebook} com processador core i3, 12 \textit{gigabytes} de memória \textit{Random Access Memory} \sigla{(RAM)}{Random Access Memory} de configurações, no qual realizou a contagem das 180 imagens com 15 \textit{megapixeis} em 14 min e 54 segundos. A diferença entre os tempos é realmente significativa, mostrando que a utilização do \textit{software} pode agregar benefícios ao laboratório de microbiologia, uma vez que além do ganho no tempo, a utilização do \textit{software} não exige conhecimento prévio, podendo ser operado por qualquer pessoa.

Considerando os resultados apresentados neste trabalho concluí-se que a abordagem baseada no brilho e saturação das imagens apresentou-se em um primeiro momento suficiente para inicio de estudo, porém é de fundamental importância enfatizar que tais resultados foram obtidos a partir de uma base de imagens pequena, e que se faz necessário estudo posteriores, com base de imagens maiores afim de verificar tais resultados aqui apresentados se fazem igualmente satisfatórios. Este trabalho de alguma forma contribui para o inicio de um trabalho mais abrangente, deixando como sugestão para trabalhos futuros o uso de técnicas de visão computacional, como \textit{Haar Cascade} e \textit{Deep Learning}.
