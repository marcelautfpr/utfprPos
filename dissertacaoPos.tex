
\documentclass[oneside]{normas_pos-utf-tex} %oneside = para dissertacoes com numero de paginas menor que 100 (apenas frente da folha) 

% force A4 paper format
%\special{papersize=210mm,297mm}

\usepackage[brazil]{babel,varioref} % pacote portugues brasileiro
\usepackage{float}
\usepackage[utf8]{inputenc} % pacote para acentuacao direta
\usepackage{amsmath,amsfonts,amssymb} % pacote matematico
\usepackage{graphicx} % pacote grafico
%\usepackage{times} % fonte times
\usepackage[final]{pdfpages} % adicao da ata
\usepackage[hidelinks]{hyperref} % gera hiperlinks para o sumario, links, referencias -- deve vir antes do 'abntcite' 
\usepackage{scalefnt}
\usepackage{textcomp}
\usepackage{tabularx}
\usepackage{pgfgantt}
\usepackage{rotating}
%\usepackage{titlesec}
\usepackage[alf,abnt-emphasize=bf,bibjustif,recuo=0cm, abnt-etal-cite=3, abnt-etal-list=0]{abntcite} %configuracao correta das referencias bibliograficas.
\usepackage{subfigure}
\usepackage{xtab}
\usepackage{nomencl}
\usepackage[portuguese,algoruled,longend,linesnumbered]{algorithm2e}
\usepackage{multirow}
\usepackage{longtable}
 %Para linhas numeradas e réguas de identação
%\usepackage{algorithm2e}
\usepackage{listings}
% Definindo novas cores
\definecolor{verde}{rgb}{0.25,0.5,0.35}
\definecolor{jpurple}{rgb}{0.5,0,0.35}
% Configurando layout para mostrar codigos Java
\lstset{
  language=Java,
  basicstyle=\ttfamily\small,
  keywordstyle=\color{jpurple}\bfseries,
  stringstyle=\color{red},
  commentstyle=\color{verde},
  morecomment=[s][\color{blue}]{/**}{*/},
  extendedchars=true,
  showspaces=false,
  showstringspaces=false,
  numbers=left,
  numberstyle=\tiny,
  breaklines=true,
  backgroundcolor=\color{cyan!10},
  breakautoindent=true,
  captionpos=b,
  xleftmargin=0pt,
  tabsize=4
}
\renewcommand{\lstlistingname}{Código}
\renewcommand\lstlistlistingname{Lista de Códigos}
\newcommand*{\noaddvspace}{\renewcommand*{\addvspace}[1]{}}
%
%\usepackage{geometry}
%\geometry{bottom=1cm}

% ---------- Preambulo ----------
\instituicao{Universidade Tecnol\'ogica Federal do Paran\'a} % nome da instituicao
%\departamento{Departamento Acad\^{e}mico de Computa\c{c}\~ao} % nome do programa
\programa{Programa de P\'{o}s-Gradua\c{c}\~{a}o em Tecnologias Computacionais para o Agroneg\'{o}cio} % área ou curso

\documento{Disserta\c{c}\~{a}o}
\nivel{Pós-Graduação}
\titulacao{Mestre} 

\titulo{{Aplica\c{c}\~{a}o do processamento de imagens e aprendizado de m\'{a}quina para controle microbiol\'{o}gico de produtos l\'{a}cteos}} % titulo do trabalho em portugues
\title{\MakeUppercase{Application of image processing and machine learning for microbiological control of dairy products}} % titulo do trabalho em ingles

\autor{Marcela Marques Barbosa} % autor do trabalho
\cita{BARBOSA, Marcela Marques} % sobrenome (maiusculas), nome do autor do trabalho

\palavraschave{Machine learning, micro-organismos, alimentos, lácteos, imagens} % palavras-chave do trabalho
\keywords{Machine learning, microorganisms, food, dairy, image} % palavras-chave do trabalho em ingles

%\comentario{\UTFPRdocumentodata\ apresentado ao \UTFPRdepartamentodata\ da \ABNTinstituicaodata\ como requisito parcial para obten\c{c}\~ao do título de ``\UTFPRtitulacaodata\ em Ciência da Computação''.}

\comentario{\UTFPRdocumentodata\ apresentada como requisito parcial para obten\c{c}\~ao do grau de \UTFPRtitulacaodata\ do \UTFPRprogramadata, \ABNTinstituicaodata. \'{A}rea de Concentra\c{c}\~{a}o: Tecnologias computacionais aplicadas \`{a} agroind\'{u}stria}

\orientador[Orientadora]{Prof\textordfeminine~Dr\textordfeminine~Saraspathy Naidoo Terroso Gama de Mendon\c{c}a} % nome do orientador do trabalho
\coorientador[Co-orientador]{Prof. Dr. Pedro Luiz de Paula Filho}
\coordenador{Prof. Dr. Cĺ\'{a}udio Leones BAzzi}

\primeiroassina{Prof. Convidado 1 \\ Instituição}
\segundoassina{Prof. Convidado 2 \\ Instituição}
\terceiroassina{\ABNTcoorientadordata \\ UTFPR - Câmpus Medianeira \\ \ABNTcoorientadorname}
\quartoassina{\ABNTorientadordata \\ UTFPR - Câmpus Medianeira \\ \ABNTorientadorname}
%\quartoassina{Prof. Me. Jorge Aikes Junior\\ UTFPR - Câmpus Medianeira}


\textoaprovacao{Esta \UTFPRdocumentodata\ foi apresentado \`{a}s 14:00h do dia 7 de novembro de 2016 como requisito parcial para a obtenção do título de \UTFPRtitulacaodata~no \UTFPRprogramadata, da \ABNTinstituicaodata, Câmpus Medianeira. O candidato foi arguido pela Banca Examinadora composta pelos professores abaixo assinados. Após deliberação, a Banca Examinadora considerou o trabalho aprovado.}

\local{Medianeira} % cidade
\data{\the\year} % ano automatico

% desativa hifenizacao mantendo o texto justificado.
% thanks to Emilio C. G. Wille
\tolerance=1
\emergencystretch=\maxdimen
\hyphenpenalty=10000
\hbadness=10000
\sloppy

\definecolor{laranjautfpr}{cmyk}{0.0, 0.2, 1.0, 0.0}
\usepackage{enumitem}
\setlist{leftmargin=2cm}
\setlist{nosep}


%---------- Inicio do Documento ----------
\begin{document}
\renewcommand{\arraystretch}{1.5}

\capa % geracao automatica da capa
\folhaderosto % geracao automatica da folha de rosto

\termodeaprovacao

%------------------------DEDICATÓRIA------------------------------

%\begin{dedicatoria}
%Dedicatória do trabalho (opcional). 
%\end{dedicatoria}

%-----------------------AGRADECIMENTOS----------------------------
%\begin{agradecimentos}
%	Para que esse momento se concretizasse, devo agradecimentos a muitas pessoas, tantas que um folha aqui seria pouca, de forma resumida tentarei agradecer a todos.

Por primeiro serei eternamente grata a própria UTFPR, que me abriu as portas me dando a oportunidade de estudar em uma das melhores instituições pública do Brasil. Agradeço não somente pelos auxílios financeiros, que diga-se de passagem foram fundamentais para que esse momento acontecesse, mas sobre tudo agradeço imensamente o carinho, o calor humano a mim dispensando em sua grande maioria pelos seus professores e servidores e trabalhadores de forma geral da UTFPR. Tenho absoluta convicção que em outro lugar não teria acontecido o mesmo.

Por segundo, quero agradecer a todos os meus professores, e aos que de uma forma ou de outra, mesmo não sendo meu professor sempre esteve do meu lado e me ensinaram tudo o que foi possível. Aqui a lista é grande, a começar pelos professores da matemática, devo muitos agradecimentos ao professor Fausto, foram anos de convivência em projeto, o professor Diego, o professor Cleverson, as professoras Neusa e Rafaela, o professor Lucas. Devo muitos agradecimentos também aos meus queridos professores da computação, Cláudio Bazzi, Cezar Angonese, Nelson Betzek, Paulo Job, Alessandra Hoffmann, Fernando Schutz, Allan Gavioli, Paulo Lopes, Márcio Matté, Evando Pessini, Hamilton Pereira, Jorge Aikes, Juliano Lamb, Arnaldo candido, Neylor Michel, Patrícia Lopez, Ricardo Sobjack, Everton Coimbra, José Airton, Silvana Mendonça, Lairton Moacir, Elias Lira. Sou e serei para sempre grata a todos vocês. 

Além dos meus queridos professores, gostaria de deixar um agradecimento aos meus companheiros de turma, que me ajudaram muito no decorrer do curso, entre todos, que não são muitos, queria deixar um agradecimento muito especial a Gabriela Michellon e a Samanta de Sousa, esta ultima, compaheira não só dos trabalhos acadêmicos mas também companheira para as horas incertas. Muito Obrigada a todas vocês.

Por terceiro quero agradecer aos responsáveis por esse projeto, a Dra Tania, o professor Dr Pedro e professora Dra Deisy que confiaram a mim esse projeto, dando-me a oportunidade de contribuir para um projeto significativo. Muito obrigada, foi uma experiência única e gratificante ter tido a honra de trabalhar com vocês.

Em especial, agradecer ao professor Dr. Arnaldo Candido pela paciência sempre dedicada a mim, não somente quanto a dúvidas relativas as disciplinas, mas também incrível ajuda nos artigos, na conclusão deste trabalho e indo além dos assuntos acadêmicos, me dando força nas horas que mais precisei. Fica aqui toda a minha gratidão e admiração pelo professor Dr. Arnaldo.

Por último e mais importante quero agradecer a Dra Marisa e ao professor Dr. Pedro Luiz, duas pessoas muito mais que especiais. Infelizmente, nunca vou conseguir agradecer na mesma proporção em que me ajudaram, pois não há como mensurar o que essas duas pessoas tão especiais fizeram e ainda fazem por mim, somente posso deixar aqui um singelo agradecimento e dizer que sempre serei extremamente grata, levando-os para sempre em meu coração.

Meu muito obrigada a todos por fazerem possível transformar um grande sonho em realidade.
%\end{agradecimentos}
%-----------------------EPÍGRAFE-----------------------------------
%\begin{epigrafe}
%Epígrafe do trabalho (opcional). 
%\end{epigrafe}

%resumo
\begin{resumo}
	O estudo e a compreensão do material genético das espécies propicia aos seres humanos compreender melhor as características das espécies para que possam ser armazenado como fonte de preservação das espécies, bem com propor melhorias genéticas que possam erradicar doenças. Na população de peixes, existe uma grande variedade em seu número de cromossomos, devido a grande variedade de peixes conhecida atualmente. Os objetivos são de identificar e separar os segmentos destacados na coloração, e através dos mesmos extrair características afim de classifica-lo. Para o desenvolvimento do software será utilizada a linguagem de programação C++, utilizando a biblioteca opensource Open Computer Vision. Será utilizado o método de similaridade e dissimilaridades, que se apropriam do conceito de distancias, Euclidiana, Manhattan, Mahalanobis, Coeficiente de Casamento Simples e Coeficiente de Jaccard.
\end{resumo}
%
%%abstract
\begin{abstract}
	For what a food be considered functional is required certify that they have a certain amount of micro-organism. This quantity is determined in the laboratory by plating containing the micro-organism in question and after a period of incubation is performed manually counting the number of bacterial colonies which are present on each plate. In this way this study aims to develop a software that be able to perform the counting of these colonies contained in petri dishes in an automated fashion. For such an algorithm was developed based on digital image processing techniques what be able to identify the petri dish within the image and thus make the count of the number of colonies present on each plate. For this algorithm was developed, was essential develop a standardization of image acquisition. From the specific hardware development and acquisition gives images, software execution, the results were compared taking the manual count as a parameter. As a result highlights is a globe correlation 0.948 compared to manual counting and a correlation of 0.8134 on a more individual analysis, thus it can be concluded based on the results obtained, as initial design study is satisfactory, it is however recommended continuation do study.
\end{abstract}

%% listas (opcionais, mas recomenda-se a partir de 5 elementos)
%Solução para o problema de numeração errada de equações, figuras e tabelas - Contribuições de Leandro Pasa UTFPR-MD 

\listadefiguras % geracao automatica da lista de figuras
%\listadetabelas % geracao automatica da lista de tabelas)
 
% para lista de figuras
\makeatletter
\renewcommand{\thefigure}{\@arabic\c@figure}
\makeatother
 
\makeatletter
\@removefromreset{figure}{chapter}
\makeatother
 
 
% para lista de tabelas
\makeatletter
\renewcommand{\thetable}{\@arabic\c@table}
\makeatother
 
\makeatletter
\@removefromreset{table}{chapter}
\makeatother
 
 
% para equações
\makeatletter
\renewcommand{\theequation}{\@arabic\c@equation}
\makeatother
 
\makeatletter
\@removefromreset{equation}{chapter}
\makeatother 

% Para eliminar, nas listas de figuras e tabelas, o espaçamento que aparece entre figuras de capítulos diferentes:

\addtocontents{lof}{\protect\noaddvspace} % para figuras
\addtocontents{lot}{\protect\noaddvspace} % para tabelas

%%%\listadequadros % adivinhe :)

%\IfFileExists{/etc/resolv.conf}{}{\listadefiguras} %Geração de lista de figuras e automáticas
%\IfFileExists{/etc/resolv.conf}{}{\listadetabelas} %Geração de lista de tabelas automáticas
%\listadesiglas % geracao automatica da lista de siglas
%%\listadesimbolos % geracao automatica da lista de simbolos
% sumario
\sumario % geracao automatica do sumario


%---------- Inicio do Texto ----------
% recomenda-se a escrita de cada capitulo em um arquivo texto separado (exemplo: intro.tex, fund.tex, exper.tex, concl.tex, etc.) e a posterior inclusao dos mesmos no mestre do documento utilizando o comando \input{}, da seguinte forma:

%\setcounter{page}{48}

\setlength{\parskip}{0.0cm}
\chapter{Introdução}\label{ch:intro}
	Testando commit
	\section{Problema}
	
	
	\section{Objetivos}
	
	\subsection{Objetivo geral}
	
	\subsection{Obetivos específicos}
	
%	\begin{itemize}
%
%	\end{itemize}
	
	\section{Justification}

	\section{Hypothesis}

%\input{pdi}
%\chapter{Materials and Methods} \label{cap:metod}

The materials related to the research will be given initially by the acquisition of the images, which will be acquired through the differential coloring method. For software development, the C ++ programming language will be used, using the Open Computer Vision (\sigla{Opencv}{Open Computer Vision }) version 3.1 library, which has implementations of algorithms for digital image processing and computational vision, together with the Qt development environment in Its opensource version, which facilitates the development of software with a graphical interface, facilitating its use by the user.

It will be used the similarity and dissimilarities method, which appropriates the concept of distances, Euclidian, Manhattan, Mahalanobis, Simple Wedding Coefficient and Jaccard Coefficient. Thus, measures of dissimilarity similarity allow the classification of the karyotype by their proximity or not of each other \cite{Genomika2015}.
%\chapter{Recursos de Hardware e Software} \label{cap:recur}

Neste capítulo serão apresentados os principais recursos de \textit{hardware} e \textit{software} utilizados nesse projeto, bem como a origem destes recursos.


\section{Recursos de Hardware} \label{sec:recurhard}

Os recursos de \textit{hardware} necessários englobam o quadricóptero, o sistema de comunicação e a estação base.

O quadricóptero pode ser divido em duas partes: estrutura física e placa de controle. Os componentes da estrutura física são:

\begin{itemize}
\item 1x Chassi de 45cm diâmetro 
\item 4x Motor brushless 5000kV 
\item 4x Hélice 9x4,7 GWS 
\item 4x \sigla{ESC}{Eletronic Speed Control} 30A 
\item 1x Bateria 4000mAh 7,4V
\end{itemize}

estes componentes foram emprestados pelo prof. Hugo Vieira, orientador desse trabalho. Também foi emprestada uma placa de controle, chamada ``KK multicopter'', porém essa é uma placa de baixo desempenho e espera-se substituí-la por uma melhor. O desejável seria construir a própria placa de controle, porém devido ao encapsulamento \sigla{SMD}{Surface Mounted Device} utilizado nos sensores MEMS, a montagem dessas placas requer o uso de equipamentos específicos, inviáveis para esse projeto.

Até o momento não há disponibilidade de nenhum dos componentes do sistema de comunicação, todos deverão ser adquiridos. Eles são:

\begin{itemize}
\item 2x Módulo transceptor de RF
\item 1x USB \textit{dongle}
\item 1x Rádio controle 4 ou mais canais
\item 1x Receptor 4 ou mais canais
\end{itemize}

A estação base é um computador, desktop ou portátil, recente, com sistema operacional Windows ou Linux. Será usado um computador próprio.


\section{Recursos de Software}

Alguns recursos de software utilizados dependerão das alternativas de hardware escolhidas e só poderão ser definidas posteriormente. Inicialmente serão utilizados os seguintes:

\begin{itemize}
\item Matlab ou Octave: simulações do sistema de controle, coleta de dados do quadricóptero. Disponíveis na UTFPR e gratuito, respectivamente.
\item Eagle: criação de diagramas eletrônicos e placas de circuito impresso. Gratuito.
\item Astah community ou Dia: edição de diagramas UML e fluxogramas. Ambos gratuitos.
\end{itemize}







%
\chapter{Viabilidade e Cronograma Preliminar} \label{cap:viabi}

Neste capítulo será avaliada a viabilidade do projeto e será apresentado um cronograma preliminar de desenvolvimento.


\section{Viabilidade}

Como descrito no capítulo anterior, os principais recursos para a elaboração do projeto foram emprestados, para os de hardware, ou são gratuitos, para os de software. Resta para aquisição apenas os componentes da comunicação e da placa controladora. O gasto estimado para aquisição dos componentes e conclusão do projeto é de 300 reais, contanto que não ocorram danos no desenvolvimento, o que é completamente viável.


\section{Cronograma Preliminar}

A Tabela \ref{tab:crono} apresenta um cronograma preliminar do desenvolvimento do \sigla{TCC}{Trabalho de Conclusão de Curso}. Na sua elaboração foi considerado que o autor continuará o desenvolvimento no semestre seguinte, junto com a disciplina de TCC 2.

\begin{table}[!htb]
\caption{Cronograma} \label{tab:crono}
\begin{center}
	\begin{tabularx}{\textwidth}{|X|c|c|}
	\hline
	\textbf{Etapa} & \textbf{Data de início} & \textbf{Data de término} \\
	\hline
	Elaboração da proposta de TCC &
	11/12/12 & 19/03/13 \\
	\hline
	Entrega da proposta de TCC &
	26/03/13 & 26/03/13 \\
	\hline
	Elaboração do plano de projeto de TCC &
	26/04/13 & 23/04/13 \\
	\hline
	Entrega do plano de projeto de TCC &
	08/05/13 & 08/05/13 \\
	\hline
	Elaboração da monografia de TCC &
	03/06/13 & 10/10/13 \\
	\hline
	Projetar e montar a estrutura física &
	03/06/13 & 16/06/13 \\
	\hline
	Modelar o sistema &
	17/06/13 & 30/06/13 \\
	\hline
	Projetar e implementar o sistema de comunicação &
	01/07/13 & 28/07/13 \\
	\hline
	Projetar e construir o sistema embarcado &
	01/07/13 & 28/07/13 \\
	\hline
	Projetar e implementar o sistema de controle &
	29/07/13 & 01/09/13 \\
	\hline
	Idealizar e conduzir experimentos reais de teste de navegação &
	02/09/13 & 06/10/13 \\
	\hline
	Entrega da monografia e  defesa do TCC &
	11/10/13 & 11/10/13 \\
	\hline	
	\end{tabularx}
\end{center}
\end{table}
%\chapter{EXPECTED RESULTS}\label{cap:Resultados}

It is expected that at the end of the whole process, the developed software will be able to identify which fish species belongs to this karyotype in the image. However, given the characteristic of the problem, it is known from the outset that it is not possible to obtain 100\% recognition of the test base, however, efforts will be devoted to identifying positively the closest of the totality. In this way, it is expected that such research will produce significant results, increasing interest in the topic addressed by this study, and encouraging further work, this time more comprehensive, to be developed

%\chapter{EXPECTED RESULTS}\label{cap:Resultados}

It is expected that at the end of the whole process, the developed software will be able to identify which fish species belongs to this karyotype in the image. However, given the characteristic of the problem, it is known from the outset that it is not possible to obtain 100\% recognition of the test base, however, efforts will be devoted to identifying positively the closest of the totality. In this way, it is expected that such research will produce significant results, increasing interest in the topic addressed by this study, and encouraging further work, this time more comprehensive, to be developed

%\chapter{EXPECTED RESULTS}\label{cap:Resultados}

It is expected that at the end of the whole process, the developed software will be able to identify which fish species belongs to this karyotype in the image. However, given the characteristic of the problem, it is known from the outset that it is not possible to obtain 100\% recognition of the test base, however, efforts will be devoted to identifying positively the closest of the totality. In this way, it is expected that such research will produce significant results, increasing interest in the topic addressed by this study, and encouraging further work, this time more comprehensive, to be developed

%\input{../meuTcc/resultados}
%
\chapter{Final Considerations} \label{cap:concl}
%\vspace{-2mm}

%Este trabalho apenas demonstra a viabilidade da contagem automatizada de UFCs de bactérias em placas de petri. Sendo apenas o inicio para estudos futuros, desta forma este trabalho apresenta apenas uma das várias possibilidades abordagem possíveis para resolução  deste problema. De forma alguma este trabalho tem a intensão de se tornar conclusivo, determinando um ponto final no estudo deste assunto. Desta maneira as conclusões aqui apresentada refere-se apenas aos resultado apresentados, sem o intuito de encerrar o assunto, ao contrário, pretende-se dar inicio a uma discussão mais detalhada sobre o assunto em questão.

Foi possível definir a padronização na aquisição das imagens que formaram a base. Esta padronização se mostrou suficiente no que diz respeito a posição, distância, iluminação e posicionamento da câmera fotográfica. Essas características permitiram a definição de um algorítimo que se mostrou eficiente para localização da placa de petri em questão, obtendo uma pequena margem de erro em poucas placas, sem comprometer nenhuma placa em sua totalidade, inclusive se mostrando flexível quanto ao deslocamento da mesma dentro da imagem. Apesar de apresentar pequenas falhas que não comprometeram a identificação da placa e nem a contagem das UFCs de bactérias, foi possível observar que alguns cuidados no ato da aquisição da imagem podem evitar alguns problemas tais como: retirar a tampa da placa de petri, a presença da tampa ocasiona brilho excessivo nas bordas da placa de petri, fazendo com que a borda tenha valores de saturação semelhantes as UFCs de bactérias, dificultando assim a identificação das mesma nessa região; Outro cuidado refere-se ao fundo utilizado, que deve conter o mínimo possível de riscos ou sujeira, evitando assim que ruídos sejam adicionados as imagens. Estes ruídos podem interferir na contagem das UFCs de bactérias pequenas, causando falsos negativos em placas com UFCs de bactérias pequenas e falsos positivos em placas com pouca quantidade de UFCs de bactérias.

Para a contagem das UFCs de bactérias, foram estudadas técnicas de processamento de digital como, algoritmos de pré-processamento das imagens, para realçar pontos de interesse e minimizar os ruídos e técnicas de segmentação que propiciassem a separação de fundo e a identificação das UFCs de bactérias. Para o devido realce das áreas de interesse foram estudadas técnicas como \textit{blur}, \textit{gaussian blur}, filtro bilateral e filtro de mediana, tendo o último se mostrado mais útil devido ao caráter variável tanto das placas de petri bem como as próprias UFCs de bactérias, por eliminar uma boa quantidade de ruídos sem comprometer as UFCs de bactérias de tamanho reduzido em placas com alta densidade de UFCs de bactérias. Para a segmentação foram estudados espaços de cores, sendo identificado que os canais de cores da família H apresentam melhores resultados, se destacando o canal HLS. Além do espaço de cores também foram utilizadas técnicas de morfologia matemática tais como: erosão; dilatação; abertura; fechamento; \textit{top hat} e \textit{black hat}. Todas essas técnicas se mostraram úteis tanto na identificação da placa de petri como a própria identificação das UFCs de bactérias. E por fim foram estudas técnicas de limiarização, aqui vale destacar seu valor ambíguo, sendo igualmente usada na fase de realce quanto na de segmentação. Para a abordagem utilizada nesse trabalho tais técnicas se mostram suficiente, pois permitiram de modo satisfatório uma segmentação possível baseando-se na saturação e luminosidade das imagens. Porém é importante ressaltar que esse conjunto de técnicas não abrangem as diversas características das imagens, por tanto, a adição de outras técnicas pode complementar este trabalho.

Dentre as possíveis abordagem para resolução do problema proposto em questão, a abordagem baseada apenas em técnicas de processamento de imagens, centrando-se na saturação e brilho das imagem foi a que se mostrou mais viável, não sendo definitiva. Tomando a contagem manual como referência, foi possível observar que a contagem automática obteve uma forte correlação em comparação com a contagem manual, segundo o cálculo de \textit{pearson} apresentado o valor de 0.9486 e erro absoluto médio de 0.2243, levando em consideração apenas os tempos por horas (média dos logs). Nesse ponto é importante destacar que esse valor não está levando em consideração a margem de erro e também é necessário considerar a baixa quantidade amostral (apenas 16 tempos distintos). Essa ressalva fica ainda mais evidenciada quando é feita uma análise mais detalhada, comparando as contagens individualmente. Para esses casos os resultados demonstram uma queda significativa no cálculo de \textit{pearson}, evidenciando uma correlação bem inferior e o erro médio absoluto apresenta valores relativamente altos.

Com tudo, é importante destacar o desempenho do \textit{software} no que se refere ao tempo. Para que se obtenha a curva de crescimento do micro-organismo são necessárias inocular e incubar 45 placas em duplicatas, somando 90 placas a serem contadas. Desta forma um técnico analista de laboratório de microbiologia qualificado e experiente pode levar mais de 8 horas de trabalho, ou seja, mais de um dia, para realizar toda a contagem e todas as 90 placas. Para que pudesse ser executado pelo \textit{software} foi obtida duas fotos de cada placa, totalizando 180 imagens e utilizou-se um \textit{notebook} com processador core i3, 12 \textit{gigabytes} de memória \textit{Random Access Memory} \sigla{(RAM)}{Random Access Memory} de configurações, no qual realizou a contagem das 180 imagens com 15 \textit{megapixeis} em 14 min e 54 segundos. A diferença entre os tempos é realmente significativa, mostrando que a utilização do \textit{software} pode agregar benefícios ao laboratório de microbiologia, uma vez que além do ganho no tempo, a utilização do \textit{software} não exige conhecimento prévio, podendo ser operado por qualquer pessoa.

Considerando os resultados apresentados neste trabalho concluí-se que a abordagem baseada no brilho e saturação das imagens apresentou-se em um primeiro momento suficiente para inicio de estudo, porém é de fundamental importância enfatizar que tais resultados foram obtidos a partir de uma base de imagens pequena, e que se faz necessário estudo posteriores, com base de imagens maiores afim de verificar tais resultados aqui apresentados se fazem igualmente satisfatórios. Este trabalho de alguma forma contribui para o inicio de um trabalho mais abrangente, deixando como sugestão para trabalhos futuros o uso de técnicas de visão computacional, como \textit{Haar Cascade} e \textit{Deep Learning}.




%---------- Referencias ----------
\clearpage % this is need for add +1 to pageref of bibstart used in 'ficha catalografica'.
\label{bibstart}
\bibliography{referencias} % geracao automatica das referencias a partir do arquivo bibliografia.bib
\label{bibend}



%\appendix{
%
%	
%\chapter{Resultado da Contagem por Placas}\label{apen:ContagemAbs}
%\section{Contagem Manual}
%\input{tabelaResultadoContagemManual}
%
%\section{Contagem Automática}
%\input{tabelaResultadoContagem}	
%	
%\chapter{Erros por Placas}\label{apen:ErroPlacas}
%\input{tabelaErrosPorPlacas}
%
%}

% --------- Ordenacao Afabetica da Lista de siglas --------
%\textbf{* Observa\c{c}\~oes:} a ordenacao alfabetica da lista de siglas ainda nao eh realizada de forma automatica, porem
% eh possivel se de realizar isto manualmente. Duas formas:
%
% ** Primeira forma)
%    A ordenacao eh feita com o auxilio do comando 'sort', disponivel em qualquer
% sistema Linux e UNIX, e tambem em sistemas Windows se instalado o coreutils (http://gnuwin32.sourceforge.net/packages/coreutils.htm)
% comandos para compilar e ordenar, supondo que seu arquivo se chame 'dissertacao.tex':
%
%      $ latex dissertacao
%      $ bibtex dissertacao && latex dissertacao
%      $ latex dissertacao
%      $ sort dissertacao.lsg > dissertacao.lsg.tmp
%      $ mv dissertacao.lsg.tmp dissertacao.lsg
%      $ latex dissertacao
%      $ dvipdf dissertacao.dvi
%
%
% ** Segunda forma)
%\textbf{Sugest\~ao:} crie outro arquivo .tex para siglas e utilize o comando \sigla{sigla}{descri\c{c}\~ao}.
%Para incluir este arquivo no final do arquivo, utilize o comando \input{arquivo.tex}.
%Assim, Todas as siglas serao geradas na ultima pagina. Entao, devera excluir a ultima pagina da versao final do arquivo
% PDF do seu documento.


%-------- Citacoes ---------
% - Utilize o comando \citeonline{...} para citacoes com o seguinte formato: Autor et al. (2011).
% Este tipo de formato eh utilizado no comeco do paragrafo. P.ex.: \citeonline{autor2011}

% - Utilize o comando \cite{...} para citacoeses no meio ou final do paragrafo. P.ex.: \cite{autor2011}



%-------- Titulos com nomes cientificos (titulo, capitulos e secoes) ----------
% Regra para escrita de nomes cientificos:
% Os nomes devem ser escritos em italico, 
%a primeira letra do primeiro nome deve ser em maiusculo e o restante em minusculo (inclusive a primeira letra do segundo nome).
% VEJA os exemplos abaixo.
% 
% 1) voce nao quer que a secao fique com uppercase (caixa alta) automaticamente:
%\section[nouppercase]{\MakeUppercase{Estudo dos efeitos da radiacao ultravioleta C e TFD em celulas de} {\textit{Saccharomyces boulardii}}
%
% 2) por padrao os cases (maiusculas/minuscula) sao ajustados automaticamente, voce nao precisa usar makeuppercase e afins.
% \section{Introducao} % a introducao sera posta no texto como INTRODUCAO, automaticamente, como a norma indica.


\end{document}
