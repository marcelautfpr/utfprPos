%\chapter{Levantamento Bibliográfico} \label{cap:funda}

%Nessa seção será descrito o estado da arte da importância e utilização de micro-organismos na produção de alimentos e do processamento digital de imagens.
%===========Texto acima está comentado temporiariamente=========================
%\section{Importância dos micro-organismos nos alimentos}
%
%	\subsection{Micro-organismos Patogênicos}
%	
%	Em países desenvolvidos onde os alimentos industrializados do ponto de vista de higiene e saúde púbica pode ser considerados seguros a ocorrência de doenças de natureza alimentar é consideravelmente significativa, como por exemplo nos Estados Unidos são estimados cerca de 24 milhões de pessoas afetadas por doenças de origem alimentar \cite{Bernadette2008}. No Brasil os dados ainda são escassos porém pode ser observado uma grande incidência de doenças de origem alimentar, assim como acontece no restante do mundo \cite{Bernadette2008}. Produtos químicos, toxinas de plantas e animais, vírus, parasitas, bactérias patogênicas e fungos toxigênicos, são agentes que quando presente nos alimentos pode provocar males a saúde animal e do ser humano \cite{Bernadette2008}.
%	
%	Os micro-organismos patogênicos que se expressam no trato gastro intestinal são conhecidos como enteropatogêncios e possuem grade importância na microbiologia alimentar.  De forma sucinta nosso processo digestivo inicia-se na boca, passa pelo esôfago, chega ao estômago onde já acontece uma pequena absorção de nutrientes e chega até o intestino delgado onde vai ocorrer a máxima absorção de nutrientes, este se divide em três partes, duodeno, jejuno e íleo. Esses micro-organismos agem de forma bem variada, com alguns preferindo se estabelecer no inicio do intestino delgado, no duodeno ou no jejuno e outros tem preferência pelo íleo (parte final do intestino delegado). Alguns são pouco invasores já outros podem até alcançar as correntes linfática e circulatória. Porém a grande maioria dos patógenos se utiliza do trato gastro intestinal como porta de entrada, podendo causar distúrbios no sistema nervoso, na corrente circulatória, no aparelho genital, no fígado entre outros. Um sintoma muito comum que esses patógenos causam é a diarréia. \cite{Bernadette2008}.
%	
%	Somente é considerado como surto de doença alimentar quando existe dois ou mais caso da doença relacionada a um único alimento. Levando em conta que um individuo se alimenta várias vezes durante o dia, é comum que qualquer doença se manisfeste logo após uma refeição, sendo assim é muito provável que venha a se fazer uma relação direta entre o alimento recém consumido com a patologia desenvolvida. Porém somente é possível caracterizar como um surto alimentar realizando um inquérito epidemiolígico com indivíduos que consumiram e com indivíduos que não consumiram o alimento. Considerando a idade, o estado nutricional, a sensibilidade, e a quantidade ingerida do alimento cada indivíduo pode apresentar uma sintomatologia distinta \cite{Bernadette2008}.
%	
%
%	\subsection{Deterioração Microbiana de Alimentos}
%	
%	Deterioração ou biodeterioração são micro-organismos que se desenvolvem no alimento causando alteração em sua composição química, organolépticas ou estrutura \cite{Bernadette2008}. Qualquer alteração que torne o alimento inapropriado para o consumo é considerada degradação. Ela pode ser provocada por bactérias, fungos, roedores, danos físicos, atividades enzimáticas \cite{Forsythe}.
%	
%	A contaminação dos alimentos ocorre na colheita, processamento e manipulação dos alimentos, e é durante a distribuição e estocagem que os micro-organismos encontram condições ideais para seu desenvolvimento ocasionado a deterioração dos alimentos. Sabe-se que a temperatura possuem grande relevância no desenvolvimento desses micro-organismos, sendo que a diminuição da temperatura pode diminuir o seu desenvolvimento. Vale ressaltar que o crescimento de micro-organismos não são a única forma de deterioração dos alimentos,a produção de metabólitos finais causam odores indesejáveis, produção de camada limosa e gás. Considerando a suscetibilidade dos alimentos em decorrência da grande variedade micro-organismos que em condições favorável podem ser desenvolver os alimentos são classificados como não-perecíveis, semiperecíveis e perecíveis. Essa classificação está relacionada a fatores internos como atividade de água, pH, agentes antimicrobianos, entre outros. Um exemplo é a farinha que é considerada um produto não-perecícvel (ou estável) pois possui uma pequena atividade de água. As maças são semiperecíveis pois pode ser favorecer o crescimento de fungos caso seja mal estocadas. E as carnes cruas que consideradas perecíveis por ter fatores internos como pH e atividade de água favorecem o crescimento de micro-organismos deteriorantes \cite{Forsythe}.
%	
%	As \textit{Pseudomonas}, as \textit{Alteromons}, as \textit{Shewanella putrefaciens} e as \textit{Aeromonos} spp, são exemplos de micro-organismos deteriorantes gram-negativos que deterioram produtos lácteos, carne vermelha, carne de galinha, peixes e ovos, pois esses alimentos possuem uma alta atividade de água, pH neutros e são estocados na presença de oxigênio em níveis normais. Os micro-organismos deteriorantes gram-positivos não formadores de esporos como as bactérias ácido-láticas causam são responsáveis pela deterioração de carnes embaladas a vácuo e também podem produzir ácido lático que causam gosto amargo em vinhos e cervejas. Os gram-positivos formadores de esporo são capazes de sobreviver em alimentos termicamente processados, podendo crescer no leite pasteurizado produzindo coalho doce e nata fina, outros causam deterioração em enlatados podendo ou não produzirem gás e estufarem as latas nesse grupo ainda existem aqueles que além de produzirem gás também produzem cheiro característico de enxofre. Mofos e leveduras deteriorantes são deteriorantes de vegetais e produtos de panificação por serem mais resistentes a baixas atividades de água e pHs ácidos, os fungos também podem deteriorar carnes \cite{Forsythe}.

%	\subsubsection{Deterioração de produtos lácteos}

%	Devido ao seu manuseio e forma de extração alguns micro-organismos já estão presentes no leite como \textit{Pseudomonoas} spp., \textit{Flavobacterium} spp., \textit{Alcaligenes} spp produzem lipase e proteases que sobrevivem a pasteurização e seu crescimento trazem um aroma e sabor rançoso desegradável ao leite. As \textit{Aeromonas} spp., \textit{Serratia} e \textit{Bacillus} spp. azedam o leite e altas contagens desses micro-organismos antes da pasteurização pode reduzir a vida de prateleira do leite. No processo de pasteurização eliminadas bactérias patógenas como \textit{Mycobacterium}, \textit{tuberculosis}, \textit{Salmonella} spp., e \textit{Brucella} spp. \cite{Forsythe}.
	
%	\subsubsection{Deterioração de produtos de carne bovina e de frango}
%	
%	A combinação de seu ph com sua alta atividade de água e por ter um valor proteico elevado as carnes são consideradas um produto altamente perecível. Bactérias ácidos-láticas, \textit{Pseudomonas} spp., quando encontradas nas carnes pode inutilizar seu consumo. Devido a possibilidade do crescimento da \textit{Salmonella} e outros patógenos os produtos cárneos merecem uma atenção maior. A pele de aves também é motivo de grande preocupação pois além de micro-organismos deteriorantes como \textit{Lactobacillos} spp., e \textit{Sh. putrefaciens} a pele também pode desenvolver micro-organismos patógenos como \textit{S. enteritidis} que podem infectar os ovários da ave podendo contaminar seus ovos \cite{Forsythe}.
%=============================O texto abaixo está comentado temporiariamente=====
%	\section{Micro-organismos Benéficos}
%		legislação(produção de derivados lácteos, iorgutes);
%		
%
%	\subsection{Utilização de Micro-organismos na produção de derivados lácteos}