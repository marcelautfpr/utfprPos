
\documentclass[oneside,english]{normas_posEn-utf-tex} %oneside = para dissertacoes com numero de paginas menor que 100 (apenas frente da folha) 

% force A4 paper format
%\special{papersize=210mm,297mm}

\usepackage[english]{babel,varioref} % pacote portugues brasileiro
\usepackage{float}
\usepackage[utf8]{inputenc} % pacote para acentuacao direta
\usepackage{amsmath,amsfonts,amssymb} % pacote matematico
\usepackage{graphicx} % pacote grafico
%\usepackage{times} % fonte times
\usepackage[final]{pdfpages} % adicao da ata
\usepackage[hidelinks]{hyperref} % gera hiperlinks para o sumario, links, referencias -- deve vir antes do 'abntcite' 
\usepackage{scalefnt}
\usepackage{textcomp}
\usepackage{tabularx}
\usepackage{pgfgantt}
\usepackage{rotating}
%\usepackage{titlesec}
\usepackage[alf,abnt-emphasize=bf,bibjustif,recuo=0cm, abnt-etal-cite=2, abnt-etal-list=0]{abntcite} %configuracao correta das referencias bibliograficas.
\usepackage{subfigure}
\usepackage{xtab}
\usepackage{nomencl}
\usepackage[portuguese,algoruled,longend,linesnumbered]{algorithm2e}
\usepackage{multirow}
\usepackage{longtable}
 %Para linhas numeradas e réguas de identação
%\usepackage{algorithm2e}
\usepackage{listings}
% Definindo novas cores
\definecolor{verde}{rgb}{0.25,0.5,0.35}
\definecolor{jpurple}{rgb}{0.5,0,0.35}
% Configurando layout para mostrar codigos Java
\lstset{
  language=Java,
  basicstyle=\ttfamily\small,
  keywordstyle=\color{jpurple}\bfseries,
  stringstyle=\color{red},
  commentstyle=\color{verde},
  morecomment=[s][\color{blue}]{/**}{*/},
  extendedchars=true,
  showspaces=false,
  showstringspaces=false,
  numbers=left,
  numberstyle=\tiny,
  breaklines=true,
  backgroundcolor=\color{cyan!10},
  breakautoindent=true,
  captionpos=b,
  xleftmargin=0pt,
  tabsize=4
}
\renewcommand{\lstlistingname}{Código}
\renewcommand\lstlistlistingname{Lista de Códigos}
\newcommand*{\noaddvspace}{\renewcommand*{\addvspace}[1]{}}
%
%\usepackage{geometry}
%\geometry{bottom=1cm}

% ---------- Preambulo ----------
\instituicao{Federal Technological University of Paran\'{a}} % nome da instituicao
%\departamento{Departamento Acad\^{e}mico de Computa\c{c}\~ao} % nome do programa
\programa{Postgraduate Program in Computational Technologies for Agribusiness} % área ou curso

\documento{Dissertation}
\nivel{Postgraduate}
\titulacao{Master} 

\titulo{{IDENTIFICATION OF FISH SPECIES: A computational approach using techniques of digital image processing and artificial intelligence}} % titulo do trabalho em portugues
\title{\MakeUppercase{IDENTIFICATION OF FISH SPECIES: A computational approach using techniques of digital image processing and artificial intelligence}} % titulo do trabalho em ingles

\autor{Marcela Marques Barbosa} % autor do trabalho
\cita{BARBOSA, Marcela Marques} % sobrenome (maiusculas), nome do autor do trabalho

\palavraschave{petri, micro-organismos, alimentos, contagem, imagens} % palavras-chave do trabalho
\keywords{petri, microorganisms, food, counting, image} % palavras-chave do trabalho em ingles

%\comentario{\UTFPRdocumentodata\ apresentado ao \UTFPRdepartamentodata\ da \ABNTinstituicaodata\ como requisito parcial para obten\c{c}\~ao do título de ``\UTFPRtitulacaodata\ em Ciência da Computação''.}

\comentario{\UTFPRdocumentodata $ \  $ presented as a partial requirement to obtain the degree of \UTFPRtitulacaodata\ of \UTFPRprogramadata, \ABNTinstituicaodata. Area of Concentration: Computational Technologies Applied to Agribusiness}

\orientador[Advisor]{Dr\textordfeminine~Saraspathy N T G De Mendonca} % nome do orientador do trabalho
\coorientador[Co-Advisor]{Prof. Dr. Pedro Luiz de Paula Filho}
\coordenador{Prof. Dr. Cláudio Leones Bazzi}


\primeiroassina{Prof. Invited 1 \\ Institution}
\segundoassina{Prof. Invited 2 \\ Institution}
\terceiroassina{\ABNTcoorientadordata \\ UTFPR - Câmpus Medianeira \\ \ABNTcoorientadorname}
\quartoassina{\ABNTorientadordata \\ UTFPR - Câmpus Medianeira \\ \ABNTorientadorname}
%\quartoassina{Prof. Me. Jorge Aikes Junior\\ UTFPR - Câmpus Medianeira}


\textoaprovacao{This \UTFPRdocumentodata ~ was presented at 14:00 on November 7, 2016 as a requirement partial in order to obtain a Master's Degree in the \UTFPRtitulacaodata~no \UTFPRprogramadata, da \ABNTinstituicaodata, C\^{a}mpus Medianeira. The candidate was accused by the examining bank composed of the professors below. After deliberation, the examining board considered the work approved.}

\local{Medianeira} % cidade
\data{\the\year} % ano automatico

% desativa hifenizacao mantendo o texto justificado.
% thanks to Emilio C. G. Wille
\tolerance=1
\emergencystretch=\maxdimen
\hyphenpenalty=10000
\hbadness=10000
\sloppy

\definecolor{laranjautfpr}{cmyk}{0.0, 0.2, 1.0, 0.0}
\usepackage{enumitem}
\setlist{leftmargin=2cm}
\setlist{nosep}


%---------- Inicio do Documento ----------
\begin{document}
\renewcommand{\arraystretch}{1.3}
\selectlanguage{english}

\capa % geracao automatica da capa
\folhaderosto % geracao automatica da folha de rosto

\termodeaprovacao


%------------------------DEDICATÓRIA------------------------------

\begin{dedicatoria}
	Dedication of work (optional).
\end{dedicatoria}

%-----------------------AGRADECIMENTOS----------------------------
\begin{agradecimentos}
	For this moment to materialize, I owe thanks to many people, so many that a sheet here would be little, in a summarized way I will try to thank everyone.

For first I will be eternally grateful to UTFPR itself, which opened the doors giving me the opportunity to study in one of the best public institutions in Brazil. I thank you not only for the financial aid, which, incidentally, was fundamental for this moment to happen, but above all I thank you immensely for the warmth, the human warmth to me dispensed in the great majority by your teachers and servants and workers in general of the UTFPR. I am quite convinced that it would not have happened the other way.

By the second, I want to thank all my teachers, and those who in one way or another, even though not being my teacher, have always been on my side and taught me everything that was possible. Here the list is great, beginning with the teachers of mathematics, I owe a lot of thanks to Professor Fausto, they were years of coexistence in design, Professor Diego, Professor Cleverson, Professors Neusa and Rafaela, Professor Lucas. Many thanks also to my beloved computer teachers, Claudio Bazzi, Cezar Angonese, Nelson Betzek, Paulo Job, Alessandra Hoffmann, Fernando Schutz, Allan Gavioli, Paulo Lopes, Márcio Matté, Evandro Pessini, Hamilton Pereira, Jorge Aikes, Juliano Lamb, Arnaldo Candido, Neylor Michel, Patrícia Lopez, Ricardo Sobjack, Everton Coimbra, José Airton, Silvana Mendonça, Lairton Moacir, Elias Lira. I am and will forever be grateful to you all.

Besides my dear teachers, I would like to thank my classmates, who helped me a lot during the course, among all, which are not many, I would like to thank Gabriela Michellon and Samanta de Sousa very much. Last, companion not only of the academic works but also companion for the uncertain hours. Thank you very much to all of you.

Thirdly, I would like to thank those responsible for this project, Dr\textordfeminine. Tania, Professor Dr Pedro and Professor Dr\textordfeminine. Deisy, who have entrusted me with this project, giving me the opportunity to contribute to a significant project. Thank you very much, it was a unique and rewarding experience to have had the honor of working with you.

In particular, I would like to thank Dr. Arnaldo Candido for his patience always dedicated to me, not only regarding doubts regarding the subjects, but also incredible help in the articles, at the conclusion of this work and going beyond the academic subjects, giving me strength in the hours that But I needed it. Here is all my gratitude and admiration for Professor Dr. Arnaldo.

%Lastly and most importantly, I want to thank Dr\textordfeminine. Marisa and Dr. Pedro Luiz, two very special people. Unfortunately, I will never be able to thank them as much as they helped me, because there is no way to measure what these two special people have done and still do for me, I can only leave here a simple thank you and say that I will always be extremely grateful, Forever in my heart.
%
%My thanks to everyone for making it possible to turn a great dream into reality.
\end{agradecimentos}
%-----------------------EPÍGRAFE-----------------------------------
\begin{epigrafe}
	Epigraph of work (optional).
\end{epigrafe}

%abstract
\begin{abstract}
	The study and understanding of the genetic material of the species enables humans to better understand the characteristics of the species so that they can be stored as a source of species preservation, as well as to propose genetic improvements that can eradicate diseases. In the fish population, there is a great variety in its number of chromosomes, due to the great variety of fish known today. The objectives are to identify and separate the highlighted segments in the coloring, and through them to extract characteristics in order to classify it. For the development of the software will be used the programming language C ++, using the open source library Open Computer Vision. It will be used the similarity and dissimilarities method, which appropriates the concept of distances, Euclidian, Manhattan, Mahalanobis, Simple Wedding Coefficient and Jaccard Coefficient.
\end{abstract}

%resumo
\begin{resumo}
	O estudo e a compreensão do material genético das espécies propicia aos seres humanos compreender melhor as características das espécies para que possam ser armazenado como fonte de preservação das espécies, bem com propor melhorias genéticas que possam erradicar doenças. Na população de peixes, existe uma grande variedade em seu número de cromossomos, devido a grande variedade de peixes conhecida atualmente. Os objetivos são de identificar e separar os segmentos destacados na coloração, e através dos mesmos extrair características afim de classifica-lo. Para o desenvolvimento do software será utilizada a linguagem de programação C++, utilizando a biblioteca opensource Open Computer Vision. Será utilizado o método de similaridade e dissimilaridades, que se apropriam do conceito de distancias, Euclidiana, Manhattan, Mahalanobis, Coeficiente de Casamento Simples e Coeficiente de Jaccard.
\end{resumo}

%% listas (opcionais, mas recomenda-se a partir de 5 elementos)
%Solução para o problema de numeração errada de equações, figuras e tabelas - Contribuições de Leandro Pasa UTFPR-MD 

\listadefiguras % geracao automatica da lista de figuras
%\listadetabelas % geracao automatica da lista de tabelas)
 
% para lista de figuras
\makeatletter
\renewcommand{\thefigure}{\@arabic\c@figure}
\makeatother
 
\makeatletter
\@removefromreset{figure}{chapter}
\makeatother
 
 
% para lista de tabelas
\makeatletter
\renewcommand{\thetable}{\@arabic\c@table}
\makeatother
 
\makeatletter
\@removefromreset{table}{chapter}
\makeatother
 
 
% para equações
\makeatletter
\renewcommand{\theequation}{\@arabic\c@equation}
\makeatother
 
\makeatletter
\@removefromreset{equation}{chapter}
\makeatother 

% Para eliminar, nas listas de figuras e tabelas, o espaçamento que aparece entre figuras de capítulos diferentes:

\addtocontents{lof}{\protect\noaddvspace} % para figuras
\addtocontents{lot}{\protect\noaddvspace} % para tabelas

%\listadefiguras % geracao automatica da lista de figuras
%\listadetabelas % geracao automatica da lista de tabelas
%%%\listadequadros % adivinhe :)

%\IfFileExists{/etc/resolv.conf}{}{\listadefiguras} %Geração de lista de figuras e automáticas
%\IfFileExists{/etc/resolv.conf}{}{\listadetabelas} %Geração de lista de tabelas automáticas
\listadesiglas % geracao automatica da lista de siglas
%%\listadesimbolos % geracao automatica da lista de simbolos
% sumario
\sumario % geracao automatica do sumario


%---------- Inicio do Texto ----------
% recomenda-se a escrita de cada capitulo em um arquivo texto separado (exemplo: intro.tex, fund.tex, exper.tex, concl.tex, etc.) e a posterior inclusao dos mesmos no mestre do documento utilizando o comando \input{}, da seguinte forma:

%\setcounter{page}{48}

\setlength{\parskip}{0cm}
\chapter{Introdução}\label{ch:intro}
	Testando commit
	\section{Problema}
	
	
	\section{Objetivos}
	
	\subsection{Objetivo geral}
	
	\subsection{Obetivos específicos}
	
%	\begin{itemize}
%
%	\end{itemize}
	
	\section{Justification}

	\section{Hypothesis}

%%\chapter{Levantamento Bibliográfico} \label{cap:funda}

%Nessa seção será descrito o estado da arte da importância e utilização de micro-organismos na produção de alimentos e do processamento digital de imagens.
%===========Texto acima está comentado temporiariamente=========================
%\section{Importância dos micro-organismos nos alimentos}
%
%	\subsection{Micro-organismos Patogênicos}
%	
%	Em países desenvolvidos onde os alimentos industrializados do ponto de vista de higiene e saúde púbica pode ser considerados seguros a ocorrência de doenças de natureza alimentar é consideravelmente significativa, como por exemplo nos Estados Unidos são estimados cerca de 24 milhões de pessoas afetadas por doenças de origem alimentar \cite{Bernadette2008}. No Brasil os dados ainda são escassos porém pode ser observado uma grande incidência de doenças de origem alimentar, assim como acontece no restante do mundo \cite{Bernadette2008}. Produtos químicos, toxinas de plantas e animais, vírus, parasitas, bactérias patogênicas e fungos toxigênicos, são agentes que quando presente nos alimentos pode provocar males a saúde animal e do ser humano \cite{Bernadette2008}.
%	
%	Os micro-organismos patogênicos que se expressam no trato gastro intestinal são conhecidos como enteropatogêncios e possuem grade importância na microbiologia alimentar.  De forma sucinta nosso processo digestivo inicia-se na boca, passa pelo esôfago, chega ao estômago onde já acontece uma pequena absorção de nutrientes e chega até o intestino delgado onde vai ocorrer a máxima absorção de nutrientes, este se divide em três partes, duodeno, jejuno e íleo. Esses micro-organismos agem de forma bem variada, com alguns preferindo se estabelecer no inicio do intestino delgado, no duodeno ou no jejuno e outros tem preferência pelo íleo (parte final do intestino delegado). Alguns são pouco invasores já outros podem até alcançar as correntes linfática e circulatória. Porém a grande maioria dos patógenos se utiliza do trato gastro intestinal como porta de entrada, podendo causar distúrbios no sistema nervoso, na corrente circulatória, no aparelho genital, no fígado entre outros. Um sintoma muito comum que esses patógenos causam é a diarréia. \cite{Bernadette2008}.
%	
%	Somente é considerado como surto de doença alimentar quando existe dois ou mais caso da doença relacionada a um único alimento. Levando em conta que um individuo se alimenta várias vezes durante o dia, é comum que qualquer doença se manisfeste logo após uma refeição, sendo assim é muito provável que venha a se fazer uma relação direta entre o alimento recém consumido com a patologia desenvolvida. Porém somente é possível caracterizar como um surto alimentar realizando um inquérito epidemiolígico com indivíduos que consumiram e com indivíduos que não consumiram o alimento. Considerando a idade, o estado nutricional, a sensibilidade, e a quantidade ingerida do alimento cada indivíduo pode apresentar uma sintomatologia distinta \cite{Bernadette2008}.
%	
%
%	\subsection{Deterioração Microbiana de Alimentos}
%	
%	Deterioração ou biodeterioração são micro-organismos que se desenvolvem no alimento causando alteração em sua composição química, organolépticas ou estrutura \cite{Bernadette2008}. Qualquer alteração que torne o alimento inapropriado para o consumo é considerada degradação. Ela pode ser provocada por bactérias, fungos, roedores, danos físicos, atividades enzimáticas \cite{Forsythe}.
%	
%	A contaminação dos alimentos ocorre na colheita, processamento e manipulação dos alimentos, e é durante a distribuição e estocagem que os micro-organismos encontram condições ideais para seu desenvolvimento ocasionado a deterioração dos alimentos. Sabe-se que a temperatura possuem grande relevância no desenvolvimento desses micro-organismos, sendo que a diminuição da temperatura pode diminuir o seu desenvolvimento. Vale ressaltar que o crescimento de micro-organismos não são a única forma de deterioração dos alimentos,a produção de metabólitos finais causam odores indesejáveis, produção de camada limosa e gás. Considerando a suscetibilidade dos alimentos em decorrência da grande variedade micro-organismos que em condições favorável podem ser desenvolver os alimentos são classificados como não-perecíveis, semiperecíveis e perecíveis. Essa classificação está relacionada a fatores internos como atividade de água, pH, agentes antimicrobianos, entre outros. Um exemplo é a farinha que é considerada um produto não-perecícvel (ou estável) pois possui uma pequena atividade de água. As maças são semiperecíveis pois pode ser favorecer o crescimento de fungos caso seja mal estocadas. E as carnes cruas que consideradas perecíveis por ter fatores internos como pH e atividade de água favorecem o crescimento de micro-organismos deteriorantes \cite{Forsythe}.
%	
%	As \textit{Pseudomonas}, as \textit{Alteromons}, as \textit{Shewanella putrefaciens} e as \textit{Aeromonos} spp, são exemplos de micro-organismos deteriorantes gram-negativos que deterioram produtos lácteos, carne vermelha, carne de galinha, peixes e ovos, pois esses alimentos possuem uma alta atividade de água, pH neutros e são estocados na presença de oxigênio em níveis normais. Os micro-organismos deteriorantes gram-positivos não formadores de esporos como as bactérias ácido-láticas causam são responsáveis pela deterioração de carnes embaladas a vácuo e também podem produzir ácido lático que causam gosto amargo em vinhos e cervejas. Os gram-positivos formadores de esporo são capazes de sobreviver em alimentos termicamente processados, podendo crescer no leite pasteurizado produzindo coalho doce e nata fina, outros causam deterioração em enlatados podendo ou não produzirem gás e estufarem as latas nesse grupo ainda existem aqueles que além de produzirem gás também produzem cheiro característico de enxofre. Mofos e leveduras deteriorantes são deteriorantes de vegetais e produtos de panificação por serem mais resistentes a baixas atividades de água e pHs ácidos, os fungos também podem deteriorar carnes \cite{Forsythe}.

%	\subsubsection{Deterioração de produtos lácteos}

%	Devido ao seu manuseio e forma de extração alguns micro-organismos já estão presentes no leite como \textit{Pseudomonoas} spp., \textit{Flavobacterium} spp., \textit{Alcaligenes} spp produzem lipase e proteases que sobrevivem a pasteurização e seu crescimento trazem um aroma e sabor rançoso desegradável ao leite. As \textit{Aeromonas} spp., \textit{Serratia} e \textit{Bacillus} spp. azedam o leite e altas contagens desses micro-organismos antes da pasteurização pode reduzir a vida de prateleira do leite. No processo de pasteurização eliminadas bactérias patógenas como \textit{Mycobacterium}, \textit{tuberculosis}, \textit{Salmonella} spp., e \textit{Brucella} spp. \cite{Forsythe}.
	
%	\subsubsection{Deterioração de produtos de carne bovina e de frango}
%	
%	A combinação de seu ph com sua alta atividade de água e por ter um valor proteico elevado as carnes são consideradas um produto altamente perecível. Bactérias ácidos-láticas, \textit{Pseudomonas} spp., quando encontradas nas carnes pode inutilizar seu consumo. Devido a possibilidade do crescimento da \textit{Salmonella} e outros patógenos os produtos cárneos merecem uma atenção maior. A pele de aves também é motivo de grande preocupação pois além de micro-organismos deteriorantes como \textit{Lactobacillos} spp., e \textit{Sh. putrefaciens} a pele também pode desenvolver micro-organismos patógenos como \textit{S. enteritidis} que podem infectar os ovários da ave podendo contaminar seus ovos \cite{Forsythe}.
%=============================O texto abaixo está comentado temporiariamente=====
%	\section{Micro-organismos Benéficos}
%		legislação(produção de derivados lácteos, iorgutes);
%		
%
%	\subsection{Utilização de Micro-organismos na produção de derivados lácteos}
%\input{pdi}
\chapter{Materials and Methods} \label{cap:metod}

The materials related to the research will be given initially by the acquisition of the images, which will be acquired through the differential coloring method. For software development, the C ++ programming language will be used, using the Open Computer Vision (\sigla{Opencv}{Open Computer Vision }) version 3.1 library, which has implementations of algorithms for digital image processing and computational vision, together with the Qt development environment in Its opensource version, which facilitates the development of software with a graphical interface, facilitating its use by the user.

It will be used the similarity and dissimilarities method, which appropriates the concept of distances, Euclidian, Manhattan, Mahalanobis, Simple Wedding Coefficient and Jaccard Coefficient. Thus, measures of dissimilarity similarity allow the classification of the karyotype by their proximity or not of each other \cite{Genomika2015}.
%\chapter{Recursos de Hardware e Software} \label{cap:recur}

Neste capítulo serão apresentados os principais recursos de \textit{hardware} e \textit{software} utilizados nesse projeto, bem como a origem destes recursos.


\section{Recursos de Hardware} \label{sec:recurhard}

Os recursos de \textit{hardware} necessários englobam o quadricóptero, o sistema de comunicação e a estação base.

O quadricóptero pode ser divido em duas partes: estrutura física e placa de controle. Os componentes da estrutura física são:

\begin{itemize}
\item 1x Chassi de 45cm diâmetro 
\item 4x Motor brushless 5000kV 
\item 4x Hélice 9x4,7 GWS 
\item 4x \sigla{ESC}{Eletronic Speed Control} 30A 
\item 1x Bateria 4000mAh 7,4V
\end{itemize}

estes componentes foram emprestados pelo prof. Hugo Vieira, orientador desse trabalho. Também foi emprestada uma placa de controle, chamada ``KK multicopter'', porém essa é uma placa de baixo desempenho e espera-se substituí-la por uma melhor. O desejável seria construir a própria placa de controle, porém devido ao encapsulamento \sigla{SMD}{Surface Mounted Device} utilizado nos sensores MEMS, a montagem dessas placas requer o uso de equipamentos específicos, inviáveis para esse projeto.

Até o momento não há disponibilidade de nenhum dos componentes do sistema de comunicação, todos deverão ser adquiridos. Eles são:

\begin{itemize}
\item 2x Módulo transceptor de RF
\item 1x USB \textit{dongle}
\item 1x Rádio controle 4 ou mais canais
\item 1x Receptor 4 ou mais canais
\end{itemize}

A estação base é um computador, desktop ou portátil, recente, com sistema operacional Windows ou Linux. Será usado um computador próprio.


\section{Recursos de Software}

Alguns recursos de software utilizados dependerão das alternativas de hardware escolhidas e só poderão ser definidas posteriormente. Inicialmente serão utilizados os seguintes:

\begin{itemize}
\item Matlab ou Octave: simulações do sistema de controle, coleta de dados do quadricóptero. Disponíveis na UTFPR e gratuito, respectivamente.
\item Eagle: criação de diagramas eletrônicos e placas de circuito impresso. Gratuito.
\item Astah community ou Dia: edição de diagramas UML e fluxogramas. Ambos gratuitos.
\end{itemize}







%
\chapter{Viabilidade e Cronograma Preliminar} \label{cap:viabi}

Neste capítulo será avaliada a viabilidade do projeto e será apresentado um cronograma preliminar de desenvolvimento.


\section{Viabilidade}

Como descrito no capítulo anterior, os principais recursos para a elaboração do projeto foram emprestados, para os de hardware, ou são gratuitos, para os de software. Resta para aquisição apenas os componentes da comunicação e da placa controladora. O gasto estimado para aquisição dos componentes e conclusão do projeto é de 300 reais, contanto que não ocorram danos no desenvolvimento, o que é completamente viável.


\section{Cronograma Preliminar}

A Tabela \ref{tab:crono} apresenta um cronograma preliminar do desenvolvimento do \sigla{TCC}{Trabalho de Conclusão de Curso}. Na sua elaboração foi considerado que o autor continuará o desenvolvimento no semestre seguinte, junto com a disciplina de TCC 2.

\begin{table}[!htb]
\caption{Cronograma} \label{tab:crono}
\begin{center}
	\begin{tabularx}{\textwidth}{|X|c|c|}
	\hline
	\textbf{Etapa} & \textbf{Data de início} & \textbf{Data de término} \\
	\hline
	Elaboração da proposta de TCC &
	11/12/12 & 19/03/13 \\
	\hline
	Entrega da proposta de TCC &
	26/03/13 & 26/03/13 \\
	\hline
	Elaboração do plano de projeto de TCC &
	26/04/13 & 23/04/13 \\
	\hline
	Entrega do plano de projeto de TCC &
	08/05/13 & 08/05/13 \\
	\hline
	Elaboração da monografia de TCC &
	03/06/13 & 10/10/13 \\
	\hline
	Projetar e montar a estrutura física &
	03/06/13 & 16/06/13 \\
	\hline
	Modelar o sistema &
	17/06/13 & 30/06/13 \\
	\hline
	Projetar e implementar o sistema de comunicação &
	01/07/13 & 28/07/13 \\
	\hline
	Projetar e construir o sistema embarcado &
	01/07/13 & 28/07/13 \\
	\hline
	Projetar e implementar o sistema de controle &
	29/07/13 & 01/09/13 \\
	\hline
	Idealizar e conduzir experimentos reais de teste de navegação &
	02/09/13 & 06/10/13 \\
	\hline
	Entrega da monografia e  defesa do TCC &
	11/10/13 & 11/10/13 \\
	\hline	
	\end{tabularx}
\end{center}
\end{table}
\chapter{EXPECTED RESULTS}\label{cap:Resultados}

It is expected that at the end of the whole process, the developed software will be able to identify which fish species belongs to this karyotype in the image. However, given the characteristic of the problem, it is known from the outset that it is not possible to obtain 100\% recognition of the test base, however, efforts will be devoted to identifying positively the closest of the totality. In this way, it is expected that such research will produce significant results, increasing interest in the topic addressed by this study, and encouraging further work, this time more comprehensive, to be developed

%\chapter{EXPECTED RESULTS}\label{cap:Resultados}

It is expected that at the end of the whole process, the developed software will be able to identify which fish species belongs to this karyotype in the image. However, given the characteristic of the problem, it is known from the outset that it is not possible to obtain 100\% recognition of the test base, however, efforts will be devoted to identifying positively the closest of the totality. In this way, it is expected that such research will produce significant results, increasing interest in the topic addressed by this study, and encouraging further work, this time more comprehensive, to be developed

%\chapter{EXPECTED RESULTS}\label{cap:Resultados}

It is expected that at the end of the whole process, the developed software will be able to identify which fish species belongs to this karyotype in the image. However, given the characteristic of the problem, it is known from the outset that it is not possible to obtain 100\% recognition of the test base, however, efforts will be devoted to identifying positively the closest of the totality. In this way, it is expected that such research will produce significant results, increasing interest in the topic addressed by this study, and encouraging further work, this time more comprehensive, to be developed

%\input{../meuTcc/resultados}

\chapter{Final Considerations} \label{cap:concl}
%\vspace{-2mm}

%Este trabalho apenas demonstra a viabilidade da contagem automatizada de UFCs de bactérias em placas de petri. Sendo apenas o inicio para estudos futuros, desta forma este trabalho apresenta apenas uma das várias possibilidades abordagem possíveis para resolução  deste problema. De forma alguma este trabalho tem a intensão de se tornar conclusivo, determinando um ponto final no estudo deste assunto. Desta maneira as conclusões aqui apresentada refere-se apenas aos resultado apresentados, sem o intuito de encerrar o assunto, ao contrário, pretende-se dar inicio a uma discussão mais detalhada sobre o assunto em questão.

Foi possível definir a padronização na aquisição das imagens que formaram a base. Esta padronização se mostrou suficiente no que diz respeito a posição, distância, iluminação e posicionamento da câmera fotográfica. Essas características permitiram a definição de um algorítimo que se mostrou eficiente para localização da placa de petri em questão, obtendo uma pequena margem de erro em poucas placas, sem comprometer nenhuma placa em sua totalidade, inclusive se mostrando flexível quanto ao deslocamento da mesma dentro da imagem. Apesar de apresentar pequenas falhas que não comprometeram a identificação da placa e nem a contagem das UFCs de bactérias, foi possível observar que alguns cuidados no ato da aquisição da imagem podem evitar alguns problemas tais como: retirar a tampa da placa de petri, a presença da tampa ocasiona brilho excessivo nas bordas da placa de petri, fazendo com que a borda tenha valores de saturação semelhantes as UFCs de bactérias, dificultando assim a identificação das mesma nessa região; Outro cuidado refere-se ao fundo utilizado, que deve conter o mínimo possível de riscos ou sujeira, evitando assim que ruídos sejam adicionados as imagens. Estes ruídos podem interferir na contagem das UFCs de bactérias pequenas, causando falsos negativos em placas com UFCs de bactérias pequenas e falsos positivos em placas com pouca quantidade de UFCs de bactérias.

Para a contagem das UFCs de bactérias, foram estudadas técnicas de processamento de digital como, algoritmos de pré-processamento das imagens, para realçar pontos de interesse e minimizar os ruídos e técnicas de segmentação que propiciassem a separação de fundo e a identificação das UFCs de bactérias. Para o devido realce das áreas de interesse foram estudadas técnicas como \textit{blur}, \textit{gaussian blur}, filtro bilateral e filtro de mediana, tendo o último se mostrado mais útil devido ao caráter variável tanto das placas de petri bem como as próprias UFCs de bactérias, por eliminar uma boa quantidade de ruídos sem comprometer as UFCs de bactérias de tamanho reduzido em placas com alta densidade de UFCs de bactérias. Para a segmentação foram estudados espaços de cores, sendo identificado que os canais de cores da família H apresentam melhores resultados, se destacando o canal HLS. Além do espaço de cores também foram utilizadas técnicas de morfologia matemática tais como: erosão; dilatação; abertura; fechamento; \textit{top hat} e \textit{black hat}. Todas essas técnicas se mostraram úteis tanto na identificação da placa de petri como a própria identificação das UFCs de bactérias. E por fim foram estudas técnicas de limiarização, aqui vale destacar seu valor ambíguo, sendo igualmente usada na fase de realce quanto na de segmentação. Para a abordagem utilizada nesse trabalho tais técnicas se mostram suficiente, pois permitiram de modo satisfatório uma segmentação possível baseando-se na saturação e luminosidade das imagens. Porém é importante ressaltar que esse conjunto de técnicas não abrangem as diversas características das imagens, por tanto, a adição de outras técnicas pode complementar este trabalho.

Dentre as possíveis abordagem para resolução do problema proposto em questão, a abordagem baseada apenas em técnicas de processamento de imagens, centrando-se na saturação e brilho das imagem foi a que se mostrou mais viável, não sendo definitiva. Tomando a contagem manual como referência, foi possível observar que a contagem automática obteve uma forte correlação em comparação com a contagem manual, segundo o cálculo de \textit{pearson} apresentado o valor de 0.9486 e erro absoluto médio de 0.2243, levando em consideração apenas os tempos por horas (média dos logs). Nesse ponto é importante destacar que esse valor não está levando em consideração a margem de erro e também é necessário considerar a baixa quantidade amostral (apenas 16 tempos distintos). Essa ressalva fica ainda mais evidenciada quando é feita uma análise mais detalhada, comparando as contagens individualmente. Para esses casos os resultados demonstram uma queda significativa no cálculo de \textit{pearson}, evidenciando uma correlação bem inferior e o erro médio absoluto apresenta valores relativamente altos.

Com tudo, é importante destacar o desempenho do \textit{software} no que se refere ao tempo. Para que se obtenha a curva de crescimento do micro-organismo são necessárias inocular e incubar 45 placas em duplicatas, somando 90 placas a serem contadas. Desta forma um técnico analista de laboratório de microbiologia qualificado e experiente pode levar mais de 8 horas de trabalho, ou seja, mais de um dia, para realizar toda a contagem e todas as 90 placas. Para que pudesse ser executado pelo \textit{software} foi obtida duas fotos de cada placa, totalizando 180 imagens e utilizou-se um \textit{notebook} com processador core i3, 12 \textit{gigabytes} de memória \textit{Random Access Memory} \sigla{(RAM)}{Random Access Memory} de configurações, no qual realizou a contagem das 180 imagens com 15 \textit{megapixeis} em 14 min e 54 segundos. A diferença entre os tempos é realmente significativa, mostrando que a utilização do \textit{software} pode agregar benefícios ao laboratório de microbiologia, uma vez que além do ganho no tempo, a utilização do \textit{software} não exige conhecimento prévio, podendo ser operado por qualquer pessoa.

Considerando os resultados apresentados neste trabalho concluí-se que a abordagem baseada no brilho e saturação das imagens apresentou-se em um primeiro momento suficiente para inicio de estudo, porém é de fundamental importância enfatizar que tais resultados foram obtidos a partir de uma base de imagens pequena, e que se faz necessário estudo posteriores, com base de imagens maiores afim de verificar tais resultados aqui apresentados se fazem igualmente satisfatórios. Este trabalho de alguma forma contribui para o inicio de um trabalho mais abrangente, deixando como sugestão para trabalhos futuros o uso de técnicas de visão computacional, como \textit{Haar Cascade} e \textit{Deep Learning}.




%---------- Referencias ----------
\clearpage % this is need for add +1 to pageref of bibstart used in 'ficha catalografica'.
\label{bibstart}
\bibliography{referencias} % geracao automatica das referencias a partir do arquivo bibliografia.bib
\label{bibend}



%\appendix{
%
%	
%\chapter{Resultado da Contagem por Placas}\label{apen:ContagemAbs}
%\section{Contagem Manual}
%\input{tabelaResultadoContagemManual}
%
%\section{Contagem Automática}
%\input{tabelaResultadoContagem}	
%	
%\chapter{Erros por Placas}\label{apen:ErroPlacas}
%\input{tabelaErrosPorPlacas}
%
%}

% --------- Ordenacao Afabetica da Lista de siglas --------
%\textbf{* Observa\c{c}\~oes:} a ordenacao alfabetica da lista de siglas ainda nao eh realizada de forma automatica, porem
% eh possivel se de realizar isto manualmente. Duas formas:
%
% ** Primeira forma)
%    A ordenacao eh feita com o auxilio do comando 'sort', disponivel em qualquer
% sistema Linux e UNIX, e tambem em sistemas Windows se instalado o coreutils (http://gnuwin32.sourceforge.net/packages/coreutils.htm)
% comandos para compilar e ordenar, supondo que seu arquivo se chame 'dissertacao.tex':
%
%      $ latex dissertacao
%      $ bibtex dissertacao && latex dissertacao
%      $ latex dissertacao
%      $ sort dissertacao.lsg > dissertacao.lsg.tmp
%      $ mv dissertacao.lsg.tmp dissertacao.lsg
%      $ latex dissertacao
%      $ dvipdf dissertacao.dvi
%
%
% ** Segunda forma)
%\textbf{Sugest\~ao:} crie outro arquivo .tex para siglas e utilize o comando \sigla{sigla}{descri\c{c}\~ao}.
%Para incluir este arquivo no final do arquivo, utilize o comando \input{arquivo.tex}.
%Assim, Todas as siglas serao geradas na ultima pagina. Entao, devera excluir a ultima pagina da versao final do arquivo
% PDF do seu documento.


%-------- Citacoes ---------
% - Utilize o comando \citeonline{...} para citacoes com o seguinte formato: Autor et al. (2011).
% Este tipo de formato eh utilizado no comeco do paragrafo. P.ex.: \citeonline{autor2011}

% - Utilize o comando \cite{...} para citacoeses no meio ou final do paragrafo. P.ex.: \cite{autor2011}



%-------- Titulos com nomes cientificos (titulo, capitulos e secoes) ----------
% Regra para escrita de nomes cientificos:
% Os nomes devem ser escritos em italico, 
%a primeira letra do primeiro nome deve ser em maiusculo e o restante em minusculo (inclusive a primeira letra do segundo nome).
% VEJA os exemplos abaixo.
% 
% 1) voce nao quer que a secao fique com uppercase (caixa alta) automaticamente:
%\section[nouppercase]{\MakeUppercase{Estudo dos efeitos da radiacao ultravioleta C e TFD em celulas de} {\textit{Saccharomyces boulardii}}
%
% 2) por padrao os cases (maiusculas/minuscula) sao ajustados automaticamente, voce nao precisa usar makeuppercase e afins.
% \section{Introducao} % a introducao sera posta no texto como INTRODUCAO, automaticamente, como a norma indica.


\end{document}
